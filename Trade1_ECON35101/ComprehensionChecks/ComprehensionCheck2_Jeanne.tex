\documentclass[10pt, final]{article}
%\usepackage[document]{ragged2e}
\usepackage[utf8]{inputenc}

\usepackage{fancyhdr}
\setlength{\headheight}{15pt}
 
\pagestyle{fancy}
\fancyhf{}
\rhead{Comprehension Check 2}
\cfoot{\thepage}

\usepackage{multirow}
\usepackage{hyperref}

\usepackage{soul}

\usepackage{enumerate}
\usepackage{amssymb}
\usepackage{amsmath, mathtools}
\usepackage{amsopn}
\usepackage{amsthm}
\usepackage{color}
\usepackage{xcolor}
\usepackage{amsfonts}
\usepackage[makeroom]{cancel}
% \usepackage{wasysym}
\usepackage[paperwidth=8.5in,left=1in,right=1in,paperheight=11.0in,top=1in, bottom=1in]{geometry}

\usepackage{tikz}
\usetikzlibrary{decorations.markings}

\usepackage{pgfplots}

\pgfplotsset{compat = 1.15}
%\pgfplotsset{scaled y ticks=false}
\usetikzlibrary{positioning}
\usepackage{mathtools}

\usepackage{listings}

\DeclarePairedDelimiter\ceil{\lceil}{\rceil}
\DeclarePairedDelimiter\floor{\lfloor}{\rfloor}

\DeclareMathOperator{\im}{im}
\DeclareMathOperator{\detr}{det}
\DeclareMathOperator{\var}{var}
\DeclareMathOperator{\cov}{cov}
\DeclareMathOperator{\Real}{Re}
\DeclareMathOperator{\sgn}{sgn}
\DeclareMathOperator{\argmax}{argmax}
\DeclareMathOperator{\vect}{vec}


% Additional commands/shortcuts to make our life easier
\newcommand{\bm}{\begin{bmatrix}}
\newcommand{\fm}{\end{bmatrix}}
\def\a{\alpha}
\def\b{\beta}
\def\g{\gamma}
\def\D{\Delta}
\def\d{\delta}
\def\z{\zeta}
\def\k{\kappa}
\def\l{\lambda}
\def\n{\nu}
\def\e{\varepsilon}
\def\r{\rho}
\def\s{\sigma}
\def\S{\Sigma}
\def\t{\tau}
\def\x{\xi}
\def\w{\omega}
\def\W{\Omega}
\def\th{\theta}
\def\p{\phi}
\def\P{\Phi}
\newcommand{\pa}{\mathcal \partial}
\newcommand{\No}{\mathcal N}



\newcommand{\hatxi}{\hat{\mathbf{x}}^i}
\newcommand{\tildexi}{\tilde{\mathbf{x}}^i}



\title{Comprehension Check 2: Productivity in the Melitz (2003) model}
\author{Jeanne Sorin\footnote{I am thankful to Chase Abram and Tom Hierons for insightful discussions and comments.}}
\date{\today}



\begin{document}

\maketitle

\begin{enumerate}[1.]
	\item \textit{Based on the primitives in section 2 of Melitz (2003), compute quantity per worker and revenue per worker for an active firm with productivity $\phi$}
	\\
	The production function is
	\begin{align*}
		l &= f + \frac{q(\phi)}{\phi}
	\end{align*}
	Therefore
	\begin{align*}
		\frac{q(\phi)}{l(\phi)} &= \frac{q(\phi)}{f + \frac{q(\phi)}{\phi}} \\
		&= \frac{Q[\rho P \phi]^{\sigma}}{f + Q [\rho P]^{\sigma}\phi^{\sigma-1}} &\text{using (3) into (2)} \\
		&= \frac{Q[\rho P]^{\sigma}}{\frac{f}{\phi^{\sigma}} + Q [\rho P]^{\sigma}\phi^{-1}}
	\end{align*}
	Similarly for 
\begin{align*}
	\frac{r(\phi)}{l(\phi)} &= \frac{R [\rho P \phi]^{\sigma-1}}{f + Q[\rho P]^{\sigma}\phi^{\sigma-1}}
\end{align*}

	\item \textit{Footnote 7 in Melitz (2003) says "Higher productivity may also be thought of as producing a higher quality variety at equal cost." Answer the questions posed in the accompanying PDF.}

	The utility function is
	\begin{align*}
		U &= \left[\int_{\omega\in\Omega} \varphi(\omega)^{\epsilon} q(\omega)^{\rho}\textrm{d}\omega\right]^{1/\rho}
	\end{align*}
	And the price index, by definition is 
	\begin{align*}
		P &= \int_0^{n}p(\omega) q(\omega) d\omega
	\end{align*}
	WTS:
	\begin{align*}
		P &= \left[ \int_{\omega\in\Omega} \varphi(\omega)^{\epsilon\sigma} p(\omega)^{1-\sigma} \textrm{d} \omega \right]^{\frac{1}{1-\sigma}}
	\end{align*}
	To do so, we find the Marshallian demand functions $q(\omega)$ as a function of $p(\omega)$ and $\varphi(\omega)$. More precisely, we solve the household's maximization problem using a Lagrangian. We notice that maximizing $U$ is equivalent to maximizing $U^\rho$, which is quite convenient as it makes algebra (much) easier :
	\begin{align*}
		L &= U^\rho - \lambda[\int_0^{n}p(\omega) q(\omega) d\omega - R]
	\end{align*}
	where R is the total household's expenses. Taking the FOC wrt $q(\omega_1)$ and $q(\omega_2)$ and their ratio
	\begin{align*}
		\rho q(\omega)^{\rho -1} \varphi(\omega)^\epsilon &= \lambda p(\omega) \\
		(\frac{q(\omega_1)}{\omega_2})^{\rho - 1} &= \frac{p(\omega_1)}{p(\omega_2)} (\frac{\varphi(\omega_2)}{\varphi(\omega_1)})^{\epsilon}
	\end{align*}
	Rearranging, multiplying by $p(\omega_1)$ and integrating wrt $\omega_1$
	\begin{align*}
		p(\omega_1) q(\omega_1) &= q(\omega_2) \frac{p(\omega_2)^\sigma \varphi(\omega_2)^{\epsilon \sigma}}{p(\omega_1)^(\sigma-1) \varphi(\omega_1)^{-\epsilon \sigma}} \\
		\int_{\omega_1} p(\omega_1) q(\omega_1) d \omega_1 &= q(\omega_2) p(\omega_2)^\sigma \varphi(\omega_2)^{\epsilon \sigma} \int_{\omega_1} p(\omega_1)^{1-\sigma} \varphi(\omega_1)^{\epsilon \sigma} d \omega_1
	\end{align*}
	Noticing that the LHS = $R$, rearranging and solving for $q(\omega_2)$
	\begin{align*}
		q(\omega_2) &= \frac{\varphi(\omega_2)^{\epsilon \sigma}}{p(\omega_2)^\sigma} R [\int_{\omega_1} p(\omega_1)^{1-\sigma} \varphi(\omega_1)^{\epsilon \sigma} d \omega_1]^{-1}
	\end{align*}
	Because P, the price index, is the true cost of living, it must be the case that $U = \frac{R}{P}$. Thus, plugging in for $q(\omega_2)$ into the definition of U, and using this equality, we get (using $\rho = \frac{\sigma - 1}{\sigma}$)
	\begin{align*}
		U &= \frac{R}{P} = \left[\int_{\omega\in\Omega} \varphi(\omega)^{\epsilon} [\frac{\varphi(\omega)^{\epsilon \sigma}}{p(\omega)^\sigma} \frac{R}{\int_{\omega_1} p(\omega_1)^{1-\sigma} \varphi(\omega_1)^{\epsilon \sigma} d \omega_1}^{\rho}\textrm{d}\omega\right]^{1/\rho} \\
		&= \frac{R}{P} = \frac{R}{\int_{\omega_1} p(\omega_1)^{1-\sigma} \varphi(\omega_1)^{\epsilon \sigma} d \omega_1} \left[\int_{\omega\in\Omega} \varphi(\omega)^{\epsilon \sigma} p(\omega)^{1-\sigma} \textrm{d}\omega\right]^{1/\rho} \\
		&= \frac{R}{[\int_{\omega\in\Omega} \varphi(\omega)^{\epsilon \sigma} p(\omega)^{1-\sigma} \textrm{d}\omega]^{1 - \frac{1}{\rho}}} \\
		&= \frac{R}{[\int_{\omega\in\Omega} \varphi(\omega)^{\epsilon \sigma} p(\omega)^{1-\sigma} \textrm{d}\omega] ^{\frac{1}{1-\sigma}}} \\
		\Rightarrow P &= [\int_{\omega\in\Omega} \varphi(\omega)^{\epsilon \sigma} p(\omega)^{1-\sigma} \textrm{d}\omega] ^{\frac{1}{1-\sigma}}
	\end{align*}
	\\
	\\
	From our marshallian demand for $q(\omega)$ plugging in for the price index we have
	\begin{align*}
		q(\omega) &= \frac{\varphi(\omega)^{\epsilon \sigma}}{p(\omega)^\sigma} \frac{R}{P^{1-\sigma}}  \\
		&= \varphi(\omega)^{\epsilon \sigma} [\frac{p(\omega)}{P}]^{-\sigma} \underbrace{\frac{R}{P}}_{Q} \\
		&= \varphi^{\epsilon \sigma}(P \rho)^\sigma Q
	\end{align*}
	Similarly
	\begin{align*}
		r(\omega) &= p(\omega) q(\omega) =  \varphi(\omega)^{\epsilon \sigma} [\frac{p(\omega)}{P}]^{1-\sigma} R \\
		&= \varphi^{\epsilon \sigma} [P \rho]^{\sigma-1} R
	\end{align*}

Finally we let the firm's production function be $l = f+q$.
Using the fact that the pricing rule is now $p(\phi) = \frac{1}{\rho} = p$\footnote{We can verify that this pricing rule holds by solving the firm's maximization problem for a firm in monopolistic competition choosing $p(\phi)$ to maximize it's profits, with constant markup $= \frac{1}{\rho}$} as the marginal cost doesn't depend on $\phi$ anymore, we plug in for $p(\omega)$ into the expression above
\begin{align*}
	r(\phi) &= \phi^{\epsilon \sigma}(\rho P)^{\sigma-1}R
\end{align*}
And profits
\begin{align*}
	\pi(\phi) &= r(\phi) - l(\phi) = r(\phi) - f - q(\phi) \\
	&= \phi^{\epsilon \sigma} [\rho P]^{\sigma} Q - \phi^{\epsilon \sigma}(\rho P)^{\sigma-1}R - f \\
	&= \phi^{\epsilon \sigma} [\rho P]^{\sigma} Q - \frac{1}{\rho}\phi^{\epsilon \sigma}(\rho P)^{\sigma}Q - f \\
	&= \phi^{\epsilon \sigma} [\rho P]^{\sigma} Q (1 - \frac{1}{\rho}) -  f \\
	&= \phi^{\epsilon \sigma} [\rho P]^{\sigma} Q (\frac{1}{1-\sigma}) - f \\
	&= \phi^{\epsilon \sigma} [\rho P]^{\sigma-1} \frac{R}{\sigma}  - f
\end{align*}




	\item \textit{For the quality variant of the model introduced in the previous question, compute quantity per worker and revenue per worker for an active firm with productivity $\phi$}
	\\
	We first need 
	\begin{align*}
		l(\phi) &= f + q(\phi) = f + \phi^{\epsilon \sigma} [\rho P]^{\sigma} Q
	\end{align*}
	Thus
	\begin{align*}
		\frac{q(\phi)}{l(\phi)} &= \frac{\phi^{\epsilon \sigma} [\rho P]^{\sigma} Q}{f + \phi^{\epsilon \sigma} [\rho P]^{\sigma} Q}
	\end{align*}
	and
	\begin{align*}
		\frac{r(\phi)}{l(\phi)} &= \frac{\phi^{\epsilon \sigma}(\rho P)^{\sigma-1}R}{f + \phi^{\epsilon \sigma} [\rho P]^{\sigma} Q}
	\end{align*}



	\item \textit{In Melitz (2003), footnote 7 says "Given the form of product differentiation, the modeling of either type of productivity difference is isomorphic." Comment on the isomorphism's implications for quantity per worker and revenue per worker.}
\\
\\
\textit{"The modeling of either type of productivity difference is isomorphic"} implies that some of the equilibrium outcomes should be the same under both modeling. Comparing questions (1) and (3) we see that for $\frac{q(\phi)}{l(\phi)}$ to be equal across the two models, we would need both $\hat{\phi}^{\epsilon \sigma} = \tilde{\phi}^\sigma$ and $\hat{\phi}^{\epsilon \sigma} = \tilde{\phi}^{\sigma-1}$ where $\hat{\phi}$ corresponds to the quantity parameter parameter from Pr. Dingel's variant (Q3) and $\tilde{\phi}$ corresponds to the productivity parameter from Melitz (Q1).
The above condition is only satisfied if under very restrictive conditions ($\phi$ constant and $=1$ would work, but this is not very interesting). 
\\
However, with regard to prices and revenues, from (1) and (3) we see that if $\hat{\phi} = \tilde{\phi}^{\frac{\sigma-1}{\epsilon \sigma}}$, then these outcomes are the same under the quantity and the quality variants. 
\\
Therefore, it seems that there is isomorphism in this model for the price / revenue variables, but not for the quantity ones. If one is interested in the quantity equilibrium, one should be careful about their modeling choice...
\end{enumerate}


\end{document}
