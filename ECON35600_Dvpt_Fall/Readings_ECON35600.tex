\documentclass[10pt, final]{article}
%\usepackage[document]{ragged2e}
\usepackage[utf8]{inputenc}

\usepackage{fancyhdr}
\setlength{\headheight}{15pt}
 
\pagestyle{fancy}
\fancyhf{}
\rhead{Global Capitalism - Jeffry A. Frieden}
\cfoot{\thepage}

\usepackage{multirow}
\usepackage{hyperref}

\usepackage{soul}

\usepackage{enumerate}
\usepackage{amssymb}
\usepackage{amsmath, mathtools}
\usepackage{amsopn}
\usepackage{amsthm}
\usepackage{color}
\usepackage{xcolor}
\usepackage{amsfonts}
\usepackage[makeroom]{cancel}
% \usepackage{wasysym}
\usepackage[paperwidth=8.5in,left=1in,right=1in,paperheight=11.0in,top=1in, bottom=1in]{geometry}

\usepackage{tikz}
\usetikzlibrary{decorations.markings}

\usepackage{pgfplots}

\pgfplotsset{compat = 1.15}
%\pgfplotsset{scaled y ticks=false}
\usetikzlibrary{positioning}
\usepackage{mathtools}

\usepackage{listings}

\DeclarePairedDelimiter\ceil{\lceil}{\rceil}
\DeclarePairedDelimiter\floor{\lfloor}{\rfloor}

\DeclareMathOperator{\im}{im}
\DeclareMathOperator{\detr}{det}
\DeclareMathOperator{\var}{var}
\DeclareMathOperator{\cov}{cov}
\DeclareMathOperator{\Real}{Re}
\DeclareMathOperator{\sgn}{sgn}
\DeclareMathOperator{\argmax}{argmax}
\DeclareMathOperator{\vect}{vec}


% Additional commands/shortcuts to make our life easier
\newcommand{\bm}{\begin{bmatrix}}
\newcommand{\fm}{\end{bmatrix}}
\def\a{\alpha}
\def\b{\beta}
\def\g{\gamma}
\def\D{\Delta}
\def\d{\delta}
\def\z{\zeta}
\def\k{\kappa}
\def\l{\lambda}
\def\n{\nu}
\def\e{\varepsilon}
\def\r{\rho}
\def\s{\sigma}
\def\S{\Sigma}
\def\t{\tau}
\def\x{\xi}
\def\w{\omega}
\def\W{\Omega}
\def\th{\theta}
\def\p{\phi}
\def\P{\Phi}
\newcommand{\pa}{\mathcal \partial}
\newcommand{\No}{\mathcal N}



\title{ECON 35600 - Economics Development - Reading Notes \\ Shawn Cole \& Michael Kremer \\ Fall 2020}
\author{Jeanne Sorin}
\date{\today}

\begin{document}

\newcommand{\hatxi}{\hat{\mathbf{x}}^i}
\newcommand{\tildexi}{\tilde{\mathbf{x}}^i}

\maketitle

\section{Financial Development and Growth (Cole)} % (fold)
\label{sec:financial_development_and_growth_}

\subsection{Do Rural Banks Matter? Evidence from the Indian Social Banking Experiment ; Burgess \& Pande, 2005} % (fold)
\label{sub:do_rural_banks_matter_evidence_from_the_indian_social_banking_experiment_burgess_&_pande_2005}

\begin{itemize}
	\item \textbf{Question}: whether state-led expansion of credit and savings facilities can reduce poverty.
	\item \textbf{Why Important / Controversial}: many believe such programs are ineffective because captured by the elite, political considerations, and worsened informal markets outcomes, on which poor people depend. Credible evidence is hard to come up with because of the nonrandom nature of these programs.
	\item \textbf{Context}: India's 1969-1990 social banking program that opened 30,000 rural bank branches. 1977 further branch licensing policy 1:4.
	\item \textbf{Empirical Strategy}: Use the 1977 policy as an instrument that increases the number of openings in unbanked rural areas.
	\item \textbf{Findings}: Branck expantion into rural unbanked locations in India significantly reduced rural poverty, in party through increased deposit mobilization and credit disbursement by banks in rural areas.
\end{itemize}



\section{New Sec} % (fold)
\label{sec:new_sec}
\begin{itemize}
	\item \textbf{Question}:
	\item \textbf{Context}:
	\item \textbf{Empirical Strategy}:
	\item \textbf{Findings}:
\end{itemize}
% section new_sec (end)

\end{document}
